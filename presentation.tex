\documentclass{beamer}

\usepackage[utf8]{inputenc}
\usepackage{default}
\usepackage{hyperref}

\usetheme{Torino}


% TO DO: Fill in URLs in 2 places

\begin{document}

\title{Authority Control at Your Library}
\author{Jane Sandberg}
\institute{Linn-Benton Community College}


\begin{frame}
 \titlepage
\end{frame}


\begin{frame}{Land Acknowledgement}
We acknowledge with respect that we are on the ancestral lands of the Osage, Missouri people, and Illinois Confederacy, whose connections to this land continue to this day.

\end{frame}


\begin{frame}{About this session}
\begin{block}{Outcome}
 All participants will leave with a manageable goal related to authority control they can implement at their own library.
\end{block}
\begin{block}<2->{What to expect}
\begin{itemize}
 \item Hands-on work with the Web client
 \item Working in groups of 2-3
 \item Interactive
\end{itemize}

\end{block}
\begin{block}<3->{Slides available}
URL ---
\end{block}


\end{frame}

\begin{frame}{Outline}
\begin{itemize}
 \item 3 activities
 \item Break
 \item Discussion: ethical dimensions of authority control
 \item Another activity
 \item Discussion: applying what you've learned today
\end{itemize}


\end{frame}

\begin{frame}{Please interrupt and yell!}

 \begin{description}
  \item[Interrupt] when something is unclear, or if you'd like to share a very relevant experience of your own.
  \item[Yell] during activities to coordinate with other groups.
 \end{description}


\end{frame}




\begin{frame}{Definition of Authority Control}
Authority control is the process of grouping
similar items together using a specific term taken from
an authority file.

\end{frame}


\begin{frame}{Reasons for authority control projects}
\begin{itemize}
 \item Improving patron discovery experience by maintaining consistent terminology
 \item Reducing the hostile terminology that your patrons encounter
 \item Future-proofing your catalog for a linked-data context
 \item Identifying low-quality and out-of-date records in your system
 \item Creating rich search data for discovery software
\end{itemize}

\end{frame}

\begin{frame}{Authority Control and Evergreen}
\begin{itemize}
 \item Galen Charlton, Chad Cluff, and Mary Jinglewski did an excellent presentation at the 2016 Evergreen Conference
 \item Slides available at \url{https://evergreen-ils.org/wp-content/uploads/2015/11/eg16-CatalogingForester_reduce.pptx}
 \item Today's preconference will build on their presentation, especially related to 
 authority work that you can do without an authority vendor.
\end{itemize}

\end{frame}



\begin{frame}{Activity 1: Identify and correct outdated terminology}
\begin{enumerate}
 \item Working with a partner, go to \url{https://egconf.missourievergreen.org/eg/staff}
 \item Click Browse the Catalog
 \item Find a bibliographic record with a subject heading ``Cookery''.
 \item Edit the record to the more current LCSH term ``Cooking''.
\end{enumerate}

\end{frame}

\begin{frame}{Control sets vs. thesauri}
 \begin{description}
  \item[Thesaurus] -- a list of authorized and variant terms.  Each MARC authority record should be part of a thesaurus.
  \item[Control set] -- a mapping of bibliographic fields to the thesauri they should match, as well as ways to control that matching process.
 \end{description}
In practice, most libraries will only use one control set.
\end{frame}

\begin{frame}{Short control set rant}
\begin{itemize}
 \item They are helpful for controlling different fields with different thesauri.
 \item Unfortunately, the MARC standard is likely to have different thesauri going on in the same field, distinguished by an indicator or \$2.  See bug 1766378.
\end{itemize}

\texttt{650\_2 \$aUnicorns in art.} will validate, even though it is a LCSH term where a MeSH heading should be.

\end{frame}


\begin{frame}{What the validate button does}
 \begin{enumerate}
  \item Finds the control set for a particular bib field.
  \item Looks for a matching term in that control set's thesauri, based on settings in the control set.
  \item If there is at least one authority heading field that matches, show a checkmark icon.
  \item Otherwise, turn the field red.
 \end{enumerate}

\end{frame}



\begin{frame}{Activity 2: Add a local authority record}

Create a thesaurus for your new authority records:
\begin{enumerate}
 \item Go to Administration $\rightarrow$ Server Administration.
 \item Choose Authority Thesauri.
 \item Create a new thesaurus just for you!  Choose any capital letter for your short code.
 Make sure your thesaurus is part of the LoC control set.
\end{enumerate}

\end{frame}

\begin{frame}{Activity 2: Add a local authority record}

Create a local authority record:
\begin{enumerate}
 \item Open up a bibliographic record.
 \item Add a 6XX field with a term that you probably won't find in any other thesaurus.  
 \item Click the Validate button and make sure it shows up in red.
 \item Click the Link icon next to the field.
 \item Click Create and Edit.
 \item In the Subj fixed field, enter the capital letter that represents your thesaurus.
 \item Click Save, then Use this Authority, then Save.
 \item Save the bib record.
 \item Hit Validate.
\end{enumerate}

\end{frame}


\begin{frame}{Activity 2: Add a local authority record}

But wait!  The Validate button never turns any field with a \$0 red.  So...
\begin{enumerate}
 \item Remove the \$0.
 \item Hit Validate again.
\end{enumerate}

\end{frame}

\begin{frame}{Limitation of the Validate button}
\begin{block}{Based on an assumption}
 Validate assumes that catalogers only work with one record at a time.
 \end{block}
 \begin{block}<2->{Working in batch}
  If you want to create links and identify unauthorized headings on a larger scale, you need to use command-line tools:
  \begin{itemize}
   \item authority\_control\_fields.pl finds matching headings and creates links.
   \item SQL queries on your handout can identify records that don't match your authority file.
  \end{itemize}


 \end{block}

\end{frame}



\begin{frame}{Activity 3: Download MARC authorities and upload}
\begin{enumerate}
 \item Download at least one MARC authority record from one of the sources on your handout.
 \item Go to Cataloging $\rightarrow$ MARC Batch Import/Export.
 \item Set Record Type to Authority Records.
 \item Check Import Non-Matching Records.
 \item Upload your file.
 \item When it's done, go to Cataloging $\rightarrow$ Manage Authorities and search for the record you imported.
\end{enumerate}
\end{frame}

\begin{frame}{Break}
See you soon
\end{frame}





\begin{frame}{Finding one's self in the catalog}
``When people walk into the library of their own free will -- when it is an act not in response to a mandatory class assignment or work project -- what are they looking for? Many times, it is themselves -- traces of self, fragments, whole stories.''
\vfill

-- De la Tierra, T. (2008). Latina lesbian subject headings: the power of naming. \emph{Radical cataloging: Essays at the front}, 94.
\end{frame}


\begin{frame}{Finding one's self in the catalog}
``Through the use of inaccurate language in the 
Library of
Congress Subject Headings
 and 
problematic classification schemes, catalogers often unwittingly contribute to the creation of 
library environments that are passively hostile to transgender users.''
\vfill

-- Roberto, K. R. (2011). Inflexible bodies: metadata for transgender identities. \emph{Journal of Information Ethics}, 20(2), 56.
\end{frame}


\begin{frame}{Professional ethical commitments}
From the IFLA Statement of International Cataloging Principles:
 \begin{description}
  \item[Common usage] -- Vocabulary used in descriptions and access should be in accord with that 
of the majority of users. 
  \item[Representation] --  Controlled forms 
of  names  of  persons, 
corporate 
bodies  and  families  should  be  based  on  the  way  an  entity 
describes itself.
  \item[Accuracy] -- Bibliographic and authority data should be an accurate portrayal of 
the entity 
described.
 \end{description}
 
 From the ALA Code of Ethics:
 \begin{description}
  \item[Privacy] -- We protect each library user's right to privacy and confidentiality.
 \end{description}
 
 
\end{frame}

\begin{frame}{Helpful strategies for reducing hostility?}
\end{frame}




\begin{frame}{Activity 4: Choose your own}
Three options:
\begin{enumerate}
 \item View an authority record in various formats
 \item Add linked data URIs to a MARC record using MARCEdit
 \item Run SQL queries on your own authority file
 \item Do a refresher of activities 1-3
\end{enumerate}
You can get the slides here: 

\end{frame}

\begin{frame}{Activity 4a: View an authority record in various formats}
 Authority records are public, but a little hard to track down:
 \begin{itemize}
  \item Browse: http://libcat.linnbenton.edu/opac/extras/browse/marcxml/ authority.subject/1/Saint Charles Mo/1/20
  \item Specific record by ID: http://libcat.linnbenton.edu/opac/extras/ supercat/retrieve/marcxml/authority/354524
  \item List of available formats: http://libcat.linnbenton.edu/opac/extras/ supercat/formats/authority
 \end{itemize}

\end{frame}

\begin{frame}{Activity 4b: Add linked data URIs to a MARC record using MarcEdit}
 \begin{enumerate}
  \item Download and install MarcEdit: \url{http://marcedit.reeset.net/downloads}
  \item Open up some bib records in the MarcEditor feature (you may need to run MarcBreaker first)
  \item Go to Tools $\rightarrow$ Linked Data Tools $\rightarrow$ Build Linked Records
  \item Move contents of \$0 to \$9 so that it doesn't mess with Evergreen's Authority Control features
  
 \end{enumerate}

\end{frame}

\begin{frame}{Activity 4c: Run SQL queries}
In your local system, run some of the queries on your handout.  Think about additional \texttt{WHERE} clauses that might be helpful for your library.
\end{frame}



\begin{frame}{Consultant time}
Get into pairs, and act as consultants to each other.  Establish what basic parameters you have for an authority control project, and fill out a worksheet that gives the bones of a project

\end{frame}


\begin{frame}{Thank you!}

\begin{block}{Special thanks...}
...to Blake G-H, who created the demo server we used today.
\end{block}

\begin{block}{Continuing our conversations}
I am available...
\begin{itemize}
 \item ...at this conference
 \item ...on email at sandbej [at] linnbenton [dot] edu
 \item ...on the evergreen-catalogers email list
 \item ...on Github at sandbergja
\end{itemize}

\end{block}



\end{frame}



\end{document}
